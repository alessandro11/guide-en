\chapter{Introcução}

Alfabeto

\begin{center}
  \begin{tabular}{ |p{1cm}|p{5cm}| }
    \hline
    Letra& Som da letra \\ \hline
    A& ei \\ 
    \hline
    B& bi \\
    \hline
    C& ci \\
    \hline
    D& di \\
    \hline
    E& i \\
    \hline
    F& éf (f mudo) \\
    \hline
    G& gi \\
    \hline
    H& heitch\\
    \hline
    I& ai \\
    \hline
    J& jei \\
    \hline
    K& quei \\
    \hline
    L& él \\
    \hline
    M& emm \\
    \hline
    N& ehn \\
    \hline
    O& ou \\
    \hline
    P& pi \\
    \hline
    Q& quiu \\
    \hline
    R& ar \\
    \hline
    S& és \\
    \hline
    T& ti \\
    \hline
    U& iu \\
    \hline
    V& vi \\
    \hline
    W& dábliu \\
    \hline
    X& éx \\
    \hline
    Y& uai \\
    \hline
    Z& zi (US) \footnote{Estados Unidos} zéd (GB) \footnote{Grã-Bretanha, Inglaterra} \\
    \hline
  \end{tabular}
\end{center}

Os números cardinal e ordinal.
\begin{center}
  \begin{tabular}{ |c|p{3cm}|p{3cm}|p{3cm}|p{3cm}| }
    \hline
    Número& Cardinal& Som Cardinal& Ordinal& Som Ordinal \\
    \hline
    1& one& uãm& first& fi(a)rst \\
    \hline
    2& two& tchu& second& sécond \\
    \hline
    3& three& tree (t mudo)& third& third \\
    \hline
    4& four& fór& fourth& fórth \\
    \hline
    5& five& faive& fifth& fifith \\
    \hline
    6& six& six& sixth& sixth \\
    \hline
    7& seven& séven& seventh& séventh \\
    \hline
    8& eight& eitch& eighth& eith \\
    \hline
    9& nine& nâine& ninth& nãinth \\
    \hline
    10& ten& tchén& thenth& tchénth \\
    \hline
    11& eleven& elévem& elenventh& elenventh \\
    \hline
    12& twelve& tuélv& twelfth& tuélvth \\
    \hline
    13& thirteen& thirthĩn& thirteenth& thirthĩnth \\
    \hline
    14& fourteen& fóurthĩn& fourteenth& fóurthĩnth \\
    \hline
    15& fifteen& fhifthĩn& fifteenth& fhifthĩnth \\
    \hline
    16& sixteen& sixthĩn& sixteenth& sixthĩnth \\
    \hline
    17& seventeen& séventhĩn& seventeenth& séventhĩnth \\
    \hline
    18& eighteen& eithĩn& eighteenth& eithĩnth \\
    \hline
    19& nineteen& nãinethĩn& nineteenth& nãinethĩnth  \\
    \hline
    20& twenty& tuenthy& twentieth& tuenth \\
    \hline
    21& twenty-one& tuenthy uãm& twenty-first& tuenthy fi(a)rst \\
    \hline
    22& 	twenty-two&     & twenty-second& \\
    \hline
    23& 	twenty-three& 	& twenty-third& \\
    \hline
    24& 	twenty-four& 	& twenty-fourth& \\
    \hline
    25& 	twenty-five& 	& twenty-fifth& \\
    \hline
    26& 	twenty-six& 	& twenty-sixth& \\
    \hline
    27& 	twenty-seven& 	& twenty-seventh& \\
    \hline
    28& 	twenty-eight& 	& twenty-eighth& \\
    \hline
    29& 	twenty-nine& 	& twenty-ninth& \\
    \hline
    30& 	thirty& 	& thirth& \\
    \hline
    31&         thirty-one&     & thirty-first& \\
    \hline
    ...& ...& ...& ...& ... \\
    \hline
    40&         forty&          & fortieth& \\
    \hline
    50&         fifty&          & fiftieth& \\
    \hline
    60&         sixty&          & sixtieth& \\
    \hline
    70&         seventy&        & seventieth& \\
    \hline
    80&         eighty&         & eightieth& \\
    \hline
    90&         ninety&         & ninieth& \\
    \hline
    100&        one/a hundred& handrede& hundredth& \\
    \hline
    101&        one hundred and one& & hundred first& \\
    \hline
    ...& ...& ...& ...& ... \\
    \hline
    200&        two hundred& & two hundreth& \\
    \hline
    ...& ...& ...& ...& ... \\
    \hline
    900&        nine hundred& & nine hundredth& \\
    \hline
    ...& ...& ...& ...& ... \\
    \hline    
    999&        nine hundred and ninety nine& & & \\
    \hline
    1000&       one/a thousand& & & \\
    \hline
  \end{tabular}
\end{center}

Pronomes do caso reto.
\begin{center}
  \begin{tabular}{ |p{2cm}|p{2cm}| }
    \hline
    Pronome EN&          Pronome PT \\
    \cline{1-2}
    \multicolumn{2}{|c|}{Singular} \\
    \cline{1-2}
    I&                   Eu \\ \hline
    You&                 Tu/Você (caso especial é usado no singular e plural). \\ \hline
    He&                  Ele \\ \hline
    She&                 Ela \\ \hline
    It&                  Ele/Ela (para coisas, objetos, animais). \\
    \cline{1-2}
    \multicolumn{2}{|c|}{Plural} \\
    \cline{1-2}
    We&                  Nós/Vós \\ \hline
    You&                 Vocês \\ \hline
    They&                Eles/Elas \\ \hline
  \end{tabular}
\end{center}

Verbo ser/estar. Deve-se observar o sujeito da frase para inferir qual
verbo ser estar deve-se usar. p.e: martelo em inglês é hammer, ele é
um objeto, logo deve vir em mente o pronome do caso reto It (ele para
coisas objetos), então usamos is para o singular.