% Pacotes usados neste documento e suas respectivas configurações

% seleção de línguas do texto (a última é a principal/default)
\usepackage[english,brazilian]{babel}

% ------------------------------------------------------------------------------
% Definição de fontes

% formato dos arquivos-fonte (utf8 no Linux e latin1 no Windows)
\usepackage[utf8]{inputenc}	% arquivos LaTeX em Unicode (UTF8)

% usar codificação T1 para ter caracteres acentuados corretos no PDF
\usepackage[T1]{fontenc}

% fonte usada no corpo do texto (descomente apenas uma)
%\usepackage{newtxtext,newtxmath}	% Times (se não tiver, use mathptmx)
%\usepackage{lmodern}			% Computer Modern (fonte clássico LaTeX)
%\usepackage{kpfonts}			% Kepler/Palatino (idem, use mathpazo)
%\renewcommand{\familydefault}{\sfdefault} % Arial/Helvética (leia abaixo)

% A biblioteca central da UFPR recomenda usar Arial, seguindo a recomendação da
% ABNT. Essa é uma escolha ruim, pois fontes sans-serif são geralmente inade-
% quados para textos longos e impressos, sendo melhores para páginas Web.
% http://www.webdesignerdepot.com/2013/03/serif-vs-sans-the-final-battle/.

% fontes usadas em ambientes específicos
\usepackage[scaled=0.9]{helvet}		% Sans Serif
\usepackage{courier}			% Verbatim, Listings, etc

% ------------------------------------------------------------------------------

% inclusão de figuras
\usepackage{graphicx}			% incluir figuras em PDF, PNG, PS, EPS
